
Every software project consists of at least 6 phases: planning, design, development, testing, deployment and maintenance. I will mainly be discussing the first two: planning and design, because this is where the high-level decision about the architecture is made. Ideally, a software architect creates the concepts and designs for software and helps to turn those concepts into plans, just as an architect designs buildings. In smaller projects, the role of software architect will usually fall to the most experienced developer. While building architects are not usually concerned with how their ideas are implemented, a software architect is involved in all stages of the development process, not just the architecture, but any high-level decisions about the tools, coding standards or platforms to be used.

The architecture of the system is probably the most important decision to be made, as it influences all subsequent stages of development and is usually quite difficult to change afterwards. Interestingly, the choice of architecture is not just about meeting all the requirements of the project, but it must also fit the style of the organisation. There is an IT theory created by computer scientist/programmer Melvin Conway in 1967 which states: ``Organisations that design systems are forced to produce designs that are copies of the communication structures of those organisations.''\cite{paper:conway:1968}. And that theory makes a lot of sense. When there is a difficult decision to make, people are always more likely to choose something they know well rather than something that might be better but with which they have no experience. Also, not all designs might be compatible with our organisational structure. For example, a monolith with a 4-month release cycle won't work for an early-stage startup where they're adding new features to their product every week and working in short iteration cycles. The same is true vice versa. Having a microservices architecture, fast iteration cycles and the ability to deploy whenever a feature is ready is great, but when applied in an organisation with a complex hierarchical structure, where simple changes take weeks to get approved, it does not even remotely exploit the benefits the architecture has to offer.

The importance of choosing the right architecture for many software projects is underestimated and influenced by new shiny trends and fancy words in the IT industry instead of being driven primarily by requirements, which can lead to unstable, inefficient and overpriced projects.