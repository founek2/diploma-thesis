\section{Modulith}
\label{section:modulith}
`Microservice architectures are all the rage these days, but what's really important for long-term maintainability is modularity. It isn't necessary to use a network boundary to create such modules.' \cite{HOW_TO_BUILD_MODULAR_MONOLITH_CONFERENCE_INTRO}

The word \textit{Modulith} is combination of words `Modular' and `Monolith'. In previous section we discussed Monolith architectures and one of them was \textit{Modular monolith} \ref{section:modular_monolith}, which basically enforces separation of logic into modules. Modulith is taking this modular approach even further by enforcing separation not just on logic, but on database/storage as well (see Figure~\ref{img:modulith_architecture}). This way the modules are truly independent on each other, and they are in full control of its data, since now modules has to communicate with others via defined interfaces (this can be done for example directly via inter-process communication or messaging). Isolated, independent modules with interface through which they communicate formulate Modulith.

Modules being fully encapsulates give ability to even run different modules on different nodes/servers and communicate with others over the network, I call this \textbf{Hybrid Modulith}, and this is basically what Microservices are in its nature are. It is an enormous evolution compared to classical Monolith in terms of scaling, since this architecture offers ability to scale just parts of the systems (modules) instead of the whole application. Slow transition to Microservices (discussed in following Section \ref{section:microservices}) is natural step, once the need for scaling arise - just by moving required module into separate service we get a hybrid of microservices and modulith. So the whole application can be build using Modulith architecture, keeping all advantages of Microservices and in the same time removing the biggest disadvantage `distributed system' with the ability to incrementally move to distributed system once it is absolutely necessary.

Modulith is a better structured Monolith with ability to scale. It still has nature of Monolith, so it is deployed as one unit, which gives confidence of matching interfaces across modules, which is something what Microservices architecture is missing, easier debugging since it is not a distributed system and modularity of Microservice architecture for long-term maintainability. To improve deployment time for larger projects, we can deploy modules independently as modern languages generally support dynamic module loading/replacement in runtime, but this adds complexity as we need to ensure that all deployed modules are compatible and perform the update atomically. 

\begin{figure}
    \centering
    \includesvg{images/modulith_architecture.svg}
    \caption{Mondulith architecture. Modules encapsulate logic and its own data. \label{img:modulith_architecture}}
\end{figure}

\subsection{Characteristics}
\subsubsection{Performance}
Running as a single process, gives this architecture near same properties concerning performance as for Monolith \ref{section:monolith:performance} with a little drawback due to its added abstraction and isolation of data storage across modules, but the modularity being huge improvement in long-term maintainability. The Modulith architecture does not scale itself, but rather presume incremental transforming into microservices when the need arises. The right question is how much do we really need our architecture to scale. When starting a project, we can have some expectation on system load, but with how dynamically today project change from nearly from day to day, it usually starts as products A and finishes as product Z, so our presumptions on start will have to be adjusted as well. With that being sad, it is nearly impossible to guess what needs to be scaled beforehand, and rather have architecture, which allows us build fast with minimal overhead and with ability to scale once we actually need it. Also, if we look on modern hardware, we can find affordable up to 128 cores servers, which can easily run our wildest applications even without getting into distributed systems. Programming languages were adapted over the years as well: NodeJS with even loop can handle easily thousands of connections in single thread, Java recently got virtual threads (virtual threads), Golang has goroutines and Rust has asynchronous programming. All of those tools allows writing effective code when dealing with IO taks, which is what most of today applications are mostly made of.

Once we have to scale this architecture, some modules are moved into separated service, heading towards microservices and the negative aspects of distributed system will start appear - mainly the network unreliability and latency, more discussed in Microservice section \ref{section:microservices:performance}.


\subsubsection{Maintainability}
This is where the Monolith has been proved to be very problematic and the Microservice shines. Enforcing modularity on architecture level turned out to be essential in long-term maintenance. In my experience all projects usually begin as beautiful things and over time got messy. In my own opinion this is primarily caused by laziness of us developers. If there is some shortcut we can use, we will and say to ourselves: `I will fix it later', which off course never happens. I am not saying this happens all the time, but there are times, when deadlines start breathing on our neck and forcing us to do things faster. So, when we remove some of those shortcuts as were in Monolith, which allowed to do anywhere anything and force modularity on architecture level as in Modulith, developers has no other choice then to do it properly, since there will be no other way. It does mean that some features will take few minutes or hours more, but this will be well invested time compared to searching for mysterious bugs in Monolith where much, much more time is spent.

\subsubsection{Sustainability}
The modularity gives big flexibility and allows for faster iterations to better adapt for changing business requirements as well as for architecture change.

\subsubsection{Testability}
Testing is very similar as for Microservice, since the whole application is split into modules, which can be tested independently. The big advantage even over Microservice is the single process nature, which allows testing using convention tools and frameworks, which were created for Monoliths.

\subsubsection{Complexity}
In terms of complexity the Modulith architecture sits somewhere between Monolith and Microservice. In its nature it is not a distributed system as Microservices, so there are no network issues and deployment process is as simple as for Monolith. Until we start running modules in separated service, we get partially distributed system, and the complexity arises. The advantages of Modulith is the incremental migration to distributed system and ability to decide whether it is actually needed, because with Microservices we have distributed system from the start even if we want/need it or not. The same apply for implementing transactions, until we start moving towards distributed system the same transaction patterns can be implemented as were for Monolith. Distributed transactions will be discussed later in Microservice section \ref{section:microservices:complexity}.